\documentclass[10pt]{jarticle}
\usepackage{float}
\usepackage{adrobo_abst}
\usepackage[dvipdfmx]{graphicx}
\usepackage{amssymb,amsmath}
\usepackage{bm}
\usepackage[superscript]{cite}
\usepackage{enumerate}
\usepackage{url}
%\usepackage[absolute]{textpos}

\renewcommand\citeform[1]{(#1)}

\begin{document}
    
    \makeatletter
    \doctype{2024年度卒業論文概要}
    \title{視覚と行動のend-to-end学習により\\経路追従行動を模倣する手法の提案}{(経路選択の成功率向上を意図したネットワークの変更と実験的評価)}
    \etitle{A proposal for an imitation method of path-tracking behavior\\by end-to-end learning of vision and action}{(Experimental evaluation of network changes\\ to improve route selection success rate)}
    
    \author{21C1011\hspace{.5zw}石黒巧}
    \eauthor{Takumi ISHIGURO}
    
    \makeatother
    
    \abstract{Recent research explores navigation using camera images.
        Okada et al. proposed a method based on end-to-end imitation learning, mimicking navigation with a metric map and enabling path following via visual input.
        Haruyama et al. extended this by combining scenario-based navigation and pathway classification, allowing robots to reach destinations.
        However, their approach was limited to scenarios in specific areas, lacking validation in diverse environments.
        This study investigates navigation feasibility in untested scenarios using an improved network and a combined online and offline training method.
        Experiments confirmed navigation success even in challenging, unverified areas.
        }
    
    \keywords{autonomouse moblie robot, end-to-end learning, navigation}
    
    \maketitle
    
    \supervisor{指導教員: 林原靖男教授}
    
    \section{緒\hspace{2zw}言}%===========================
    近年,ロボットにおけるカメラ画像を用いたナビゲーションの研究が進んでいる.
    本研究室の岡田ら\cite{okada2020}はメトリックマップベースの経路追従行動を end-to-end 学習を用いて,模倣学習することで,視覚に基づくナビゲーション手法を提案した.
    また,春山ら\cite{haruyama2023}はカメラ画像とシナリオに基づいて,目的地まで経路追従するシステムを提案している.
    % シナリオとは島田らが提案した,「条件」と「行動」に関する単語を組みわせて構成されている.
    この手法は岡田らの手法をベースに,カメラ画像から通路の種類を分類し,シナリオに基づいて経路を選択する機能を追加している.
    % 春山らは、島田が作成した 50 例のシナリオから 7 例を選定している。
    春山らは,島田ら\cite{Shimada2020}が作成したシナリオから 7 例を選定し,目的地まで到達できることを確認している.
    シナリオの選定理由の1つに,壁や床の色が類似しており,一貫性のある環境を対象としている.
    選定外のシナリオでは,ホワイエを通過する必要があることや,地面の色が異なるエリアも対象となっているため,提案手法で自律走行できない可能性がある.
    
    そこで本論文では,島田らが作成したシナリオの中で,春山らが検証していないシナリオについて,目的地まで経路追従できるかを確認する.
    また,経路追従の成功率を向上させるための改良も行う.
    \section{視覚に基づいて目的地まで\\経路追従するシステム}
    春山らの手法\cite{haruyama2023}の概要を\reffig{system}に示す.
    システムは,
    \begin{enumerate}
        \setlength{\parskip}{0cm} % 段落間
        \setlength{\itemsep}{0cm} % 項目間
        \item シナリオを分解し,「条件」と「行動」を抽出するシナリオモジュール
        \item カメラ画像と目標方向を与えることで,経路を追従する経路追従モジュール
        \item カメラ画像から通路の特徴を分類する通路分類モジュール
    \end{enumerate}
    の 3 つから構成される.
    まずは人間が目的地に応じた「条件」と「行動」のシナリオを作成する.
    シナリオによってシナリオモジュールが指示を抽出し,経路追従モジュールが行動を実行する.
    通路分類モジュールが条件達成を検出し,次の行動に遷移する.
    \begin{figure*}[t]
        \centering
        \includegraphics[width=105mm]{./fig/haruyama/system.pdf}
        \caption{Overview of proposed system by haruyama and others(Quoted from \cite{haruyama2023})}
        \label{fig:system}
    \end{figure*}

    % ロボットは下記の a から d の一連の流れにより,指示された経路に沿って目的地まで自律移動する.
    % \begin{enumerate}
    %     \setlength{\parskip}{0cm} % 段落間
    %     \setlength{\itemsep}{0cm} % 項目間
    %     \item [(a)] 目的地に応じた「条件」と「行動」のシナリオを人間が作成.
    %     \item [(b)] シナリオモジュールが「条件」と「行動」を抽出し、初めの行動を経路追従モジュールに指示
    %     \item [(c)] 通路分類モジュールが条件達成を検出し、シナリオモジュールが次の行動へ遷移を判断
    %     \item [(d)] 経路追従モジュールが次の行動を実行
    % \end{enumerate}


    \section{機能の改善}%===========================
    経路追従モジュールに関して,経路追従の可能性を向上させるために2点変更を加えた.
    \subsection{ネットワークの変更}
    Felipeら\cite{Codevilla2018}の先行研究を基に,\reffig{branched}に示すコマンドでモデルを分岐する形式のネットワークを構築した.
    % ネットワークの入力は春山らが作成したものと同様で 64×48 の RGB画像と\tabref{tab:cmd_dir}に示す,目標方向のワンホットベクトルで,出力はヨー方向の角速度である.
    % 損失関数や活性化関数などのパラメータは春山らの手法と同様である.
    \begin{center}
        \begin{figure}[h]
            \includegraphics[width=0.475\textwidth]{./fig/ishiguro/branched.pdf}
            \caption{Branched network}
            \label{fig:branched}
        \end{figure}
    \end{center}

    \subsection{オフライン学習の併用}
    学習量を増やすため事前に取得したデータを使用し追学習を行う.

    \section{実験}%===========================
    新たなエリアを含んだシナリオに関しても,ロボットが目的地へ到達可能であるか検証する.
    \subsection{実験装置}
    \reffig{gamma}に示すロボットを使用する.
    また,実験は千葉工業大学津田沼キャンパス 2 号館 3 階で行う.
    
    \begin{figure}[h]
        \centering
        \includegraphics[width=0.275\textwidth]{./fig/ishiguro/gamma.pdf}
        \caption{Experimental setup}
        \label{fig:gamma}
    \end{figure}
    
    \subsection{実験方法}
    % \begin{center}
    %     \begin{figure}[!b]
    %         \includegraphics[width=0.45\textwidth]{./fig/ishiguro/topo.pdf}
    %         \caption{aaaaaaaaaaaa}
    %         \label{fig:route}
    %     \end{figure}
    % \end{center}

    訓練時にはオンライン学習を行いながら,すべての三叉路に侵入,脱出を行うルートを 1 周走行する.
    次に作成したモデルに追加でオフライン学習を行う.
    オフライン学習のデータセットにはオンライン学習の際に作成したルート 1 周分のデータを使用し,epoch数は 20 とした.
   
    % \subsubsection{通路分類モジュールの訓練}
    % % \figref{fig:route}に示すルートをROS の navigation パッケージを使用して,経路を 1 周する.
    % その際, 3 つのカメラからそれぞれ画像データを収集しながら走行する.
    % 学習時のパラメータとして,バッチサイズを 32 ,epoch数を 30 とした.

    訓練後,ロボットが目的地まで到達できるか確認する.
    実験では,ロボットをシナリオのスタート地点,向きに配置し,シナリオを 1 例ずつ入力する.
    途中で壁に衝突や,経路の選択を誤ることなく,目的地で停止した際に成功とする.

    \subsection{実験結果}
    シナリオ 28 例中,24 例の成功を確認した.
    失敗した場合では曲がり角で左折できなかった.
    原因として,通路分類の結果の切り替わりが遅いことが考えられる.
    目標方向を与えるタイミングが学習時より遅いため,左折できないことを確認した.

    \section{結\hspace{2zw}言}%===========================
    春山らの先行研究では走行が確認されていないシナリオでも目的地まで経路追従できることを確認した.

    \vspace{5truemm}
    {\footnotesize
        \begin{thebibliography}{99}
            
            \bibitem{okada2020}
            岡田眞也 , 清岡優祐 , 上田隆一 , 林原靖男 .
            ``視覚と行動の end-to-end 学習により経路追従行動をオンラインで模倣する手法の提案'',
            計測自動制御学会 SI 部門講演会 SICE-SI2020予稿集 , 
            pp.1147-1152(2020).

            \bibitem{haruyama2023}
            春山健太 , 藤原柾 , 馬場琉生 , 石黒巧 , 上田隆一 , 林原靖男 .
            ``視覚と行動のend-to-end学習により経路追従行動をオンラインで模倣する手法の提案 -トポロジカルマップとシナリオに基づく経路選択機能の追加と検討-'',
            計測自動制御学会 SI 部門講演会 SICE2023 予稿集,
            pp.1B4-03(2023).
       
            \bibitem{Codevilla2018}
            F. Codevilla, M. Müller, A. López, V. Koltun, A. Dosovitskiy: 
            ``End-to-end Driving via Conditional Imitation Learning'', 
            arXiv preprint, arXiv:1710.02410 (2018), 
            \url{https://arxiv.org/abs/1710.02410}.

            \bibitem{Shimada2020}
            島田滉己,上田隆一,林原靖男,
            ``トポロジカルマップを用いたシナリオによるナビゲーションの提案 -シナリオに基づく実ロボットのナビゲーション-'',
            計測自動制御学会 SI 部門講演会 SICE2020 予稿集,
            pp.1H2-04(2020).



        \end{thebibliography}
    }
    \normalsize
    
\end{document}
