\documentclass[10pt]{jarticle}
\usepackage{float}
\usepackage{adrobo_abst}
\usepackage[dvipdfmx]{graphicx}
\usepackage{amssymb,amsmath}
\usepackage{bm}
\usepackage[superscript]{cite}
\usepackage{enumerate}
\usepackage{url}
%\usepackage[absolute]{textpos}

\renewcommand\citeform[1]{(#1)}

\begin{document}
    
    \makeatletter
    \doctype{2024年度卒業論文概要}
    \title{視覚と行動のend-to-end学習により\\経路追従行動を模倣する手法の提案}{(経路選択の成功率向上を意図したネットワークの変更と実験的評価)}
    \etitle{A proposal for an imitation method of path-tracking behavior\\by end-to-end learning of vision and action}{(Experimental evaluation of network changes\\ to improve route selection success rate)}
    
    \author{21C1011\hspace{.5zw}石黒巧}
    \eauthor{Takumi ISHIGURO}
    
    \makeatother
    
    \abstract{When preparing the manuscript, read and observe carefully this sample as well as the instruction manual for the manuscript of the Transaction of Japan Society of Mechanical Engineers. This sample was prepared using MS-word. Character size of the English title is 14 pts of Times New Roman as well as sub-title. The name is 12 pts. The address of the first author and the abstract is 10 pts of Times New Roman. Character spacing of the abstract is narrowed by 0.2 pts preferably.}
    
    \keywords{Mechanical Engineering, Keywords List}
    
    \maketitle
    
    \supervisor{指導教員: 林原靖男教授}
    
    \section{緒\hspace{2zw}言}%===========================
    近年、自律移動ロボットの研究においてカメラ画像を用いたナビゲーションに関する研究が行われている。
    本研究室の岡田らはメトリックマップベースの経路追従行動を end-to-end 学習を用いて,視覚を入力として模倣することで、視覚に基づくナビゲーション手法を提案した.
    また,春山らはカメラ画像とシナリオに基づいて,任意の目的地まで自律移動するシステムを提案している.
    ここでのシナリオとは島田らが提案した,「条件」と「行動」に関する単語を組みわせて構成されている.
    この手法では,岡田らの視覚に基づいたナビゲーションに加え,カメラ画像から分岐路を認識,シナリオによって目標方向を決定し,経路を選択する機能を追加している.
    % 春山らは、島田が作成した 50 例のシナリオから 7 例を選定している。
    春山らは、島田が作成した 50 例のシナリオから 7 例を選定している。
    選定理由の1つとして、島田らが対象とするエリアから限定している。
    限定されたエリアは、ホワイエと呼ばれるスペースを一部を含むものの、壁や床の色が類似しており、一貫性のある環境といえる。
    一方、その他のエリアを含むシナリオでは、ホワイエを通り抜ける必要があることや,地面の色が異なる区域も対象としていることから,提案する手法で自律走行できないおそれがある.
    
    \section{視覚に基づいて目的地まで\\経路追従するシステム}
    \section{機能の改善}%===========================

    
    \section{実験}%===========================
        
    \section{結\hspace{2zw}言}%===========================
    
    \vspace{5truemm}
    {\footnotesize
        \begin{thebibliography}{99}
            
            \bibitem{工大2005}
            工大太郎: ``ロボットのしくみ'', 
            日本機械学会論文誌A, 
            Vol.~108, No.~1034 (2005), pp.~1--2.
            
            \bibitem{Shibutani2004}
            Y. Shibutani: ``Heinrich's Law Resulted Pattern Dynamics --Part2--'',
            Proceedings of the 79th Kansai Branch Regular Meeting of the Japan Society of Mechanical Engineers,  
            No.~04--05 (2004), pp.~205--206.
            
            \bibitem{Handbook1979}
            The Japan Society of Mechanical Engineers ed.: ``JSME Date Handbook: Heat Transfer'', 
            (1979), p.~123, The Japan Society of Mechanical Engineers.
            
            \bibitem{Kikuchi2017}
            K. Kikuchi, M. Miura, K. Shibata, J. Yamamura: ``Soft Landing Condition for Stair-climbing Robot with Hopping Mechanism'', 
            Journal of JSDE, Vol.~53, No.~8 (2018), pp.~605--614, \url{https://doi.org/10.14953/jjsde.2017.2774}.
            
            \bibitem{Adrobo2019}
            千葉工業大学 未来ロボティクス学科 学科概要: 
            \url{http://www.robotics.it-chiba.ac.jp/ja/subject/index.html}, 
            (参照日 2023年1月29日). 
            
        \end{thebibliography}
    }
    \normalsize
    
\end{document}
